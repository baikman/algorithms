\documentclass[12pt]{article}
\newcommand{\ul}[1]{\underline{#1}}
\usepackage{amsmath}

\setlength{\textwidth}{6.5in}
\setlength{\textheight}{9.0in}
\setlength{\oddsidemargin}{-0.25in}
\setlength{\evensidemargin}{-0.25in}
\setlength{\topmargin}{-0.75in}

\newcommand{\tab}{\hspace*{1em}}

\begin{document}
\noindent{\large\bf CS-3410 \hfill Algorithms \hfill Spring 2026} \\
Homework 1 \hfill {\bf Brandon Aikman}

% Homework 1 

\begin{enumerate}
	% 1.
	\item Problem 1-1 (p15)
	\begin{table}[h]
		\centering
		\caption{Comparison of Running Times}
    	\label{table:sample}
    	\begin{tabular}{|c|c|c|}
        	\hline
        	$f(n)$ & \textbf{1 second} & \textbf{1 hour} \\ \hline
        	$\log n$ & $2^{1,000,000}$ & $2^{3,600,000,000}$ \\ \hline
        	$\sqrt{n}$ & $1 \cdot 10^{12}$ & $1.296 \cdot 10^{19}$ \\ \hline
        	$n$ & $1 \cdot 10^{6}$ & $3.6 \cdot 10^{9}$ \\ \hline
        	$n \log n$ & 62,746 & $1.334 \cdot 10^{8}$ \\ \hline
        	$n^2$ & 1,000 & 60,000 \\ \hline
        	$n^3$ & 100 & 1,532 \\ \hline
        	$2^n$ & 20 & 31 \\ \hline
        	$n!$ & 9 & 12 \\ \hline
    	\end{tabular}
	\end{table}
	
	% 2.
	\item Exercise 2.3-4 (p44) \\
	Prove that when $n \geq 2$ is an exact power of 2, the solution of the recurrence 
	$$T(n) = \begin{cases}
		2 & n = 2, \\
		2T(n/2) + n & n > 2
		\end{cases}
	$$
	is $T(n) = n \log n$.
	
	% 3.
	\item Problem 2-3 (p46)
	\begin{enumerate}
		\item $\Theta (n)$
		\item 
			\texttt{naiveHorner(A,n,x)\\
			\tab p = 0\\
			\tab for i = 0 to n\\
			\tab\tab item = A[i]\\
			\tab\tab for j = 1 downto i\\
			\tab\tab\tab item *= x\\
			\tab\tab p += item\\
			\tab return p\\
			}
			The running time of \texttt{naiveHorner} is $\Theta (n^2)$, and is beat by \texttt{Horner} asymptotically.
		\item 
	\end{enumerate}
	
	% 4.
	\item Exercise 3.2-2 (p62) \\\\
	\textbf{The statement }\textit{``The running time of algorithm $A$ is at least $O(n^2)$"}\textbf{ is meaningless, because $O(n^2)$ defines an upper bound on the asymptotic behavior of the algorithm, so it will never have a time complexity greater than $O(n^2)$.}
	
	% 5.
	\item Exercise 3.2-6 (p63) \\
	Prove that $o(g(n)) \cap \omega (g(n))$ is the empty set.\\\\
	$o$-notation denotes an asymptotically loose upper bound, formally defined as the set \\\\
	$o(g(n)) = \{ f(n): \mbox{ for any positive constant }c > 0, \mbox{ there exists a constant } n_0 \mbox{ such that }\\ 0 \leq f(n) < cg(n) \mbox{ for all } n \geq n_0.\}$\\
	
	Additionally, $\omega$-notation denotes an asymptotically loose lower bound, formally defined as the set \\\\
	$\omega (g(n)) = \{f(n): \mbox{ for any positive constant }c > 0, \mbox{ there exists a constant } n_0 \mbox{ such that }\\ 0 \leq cg(n) < f(n) \mbox{ for all } n \geq n_0.\}$\\
	
	\textbf{So, we clearly see that the intersection of these two sets is the empty set, because $f(n)$ cannot be both exclusively less than $cg(n)$ and exclusively greater than $cg(n)$ simultaneously.}
	
	% 6.
	\item Using the substitution method, show that the solution of $T(n) = T(\lceil n/2 \rceil) + 1$ is $O(\log n)$. \\
	
	% 7.
	\item Exercise 4.5-1 (a, b, d, e) (p106) \\
	% 8.
	\item Exercise 4.5-2 (p106) \\
	
	% 9.
	\item Exercise 5.2-1 (p133) \\
	
	% 10.
	\item Exercise 5.2-2 (p133) \\
	
	% EC.
	\item (EC) Problem 4-6 (p122) \\
    
\end{enumerate}

\end{document}
