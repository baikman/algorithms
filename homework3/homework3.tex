\documentclass[12pt]{article}
\newcommand{\ul}[1]{\underline{#1}}
\usepackage{graphicx}
\usepackage{amsmath}

\setlength{\textwidth}{6.5in}
\setlength{\textheight}{9.0in}
\setlength{\oddsidemargin}{-0.25in}
\setlength{\evensidemargin}{-0.25in}
\setlength{\topmargin}{-0.75in}

\newcommand{\tab}{\hspace*{1em}}

\begin{document}
\noindent{\large\bf CS-3410 \hfill Algorithms \hfill Spring 2026} \\
Homework 3 	 \hfill {\bf Brandon Aikman}

% Homework 3

\begin{enumerate}
	% 1.
	\item Exercise 13.2-2 (p337)\\
		\textbf{A rotation in a tree is defined on a parent-child edge, where a left rotation is possible when a node has a right child, and a right rotation when it has a left child. So, each edge in a tree corresponds to one possible rotation. Because we see that an $n$-node tree has $n-1$ edges, it has exactly $n-1$ possible rotations.}
		
	% 2.
	\item Exercise 13.3-1 (p346)\\
		\textbf{If we set $z$'s color to black, we violate property 5: for each node, all simple paths from the node to descendant leaves contain the same number of black nodes.}
	% 3.
	\item Exercise 13.3-2 (p346)\\
	\begin{figure}[h!tbp]
    	\centering
    	\includegraphics[width=0.4\textwidth]{red-black-tree.png}
    	\caption{Red-Black Tree Insertions}
	\end{figure}\\
	
	% 4.
	\item Exercise 13.4-4 (p354)\\
	\begin{figure}[h!tbp]
    	\centering
    	\includegraphics[width=0.4\textwidth]{red-black-delete.png}
    	\caption{Red-Black Tree Deletes}
	\end{figure}

	% 5.
	\item Exercise 17.1-1 (p485)\\
	\texttt{
		key: 26; i = 10; r = 13\\
		key: 17; i = 10; r = 8\\
		key: 21; i = 2; r = 3\\
		key: 19; i = 2; r = 1\\
		key: 20; i = 1; r = 1 $\Rightarrow$ return 20
	}\\

	% 6.
	\item Exercise 17.1-2 (p485)\\
	\texttt{
		x.key: 35\\
		r = 1$\rightarrow$3$\rightarrow$16\\
		y.key: 35$\rightarrow$38$\rightarrow$30$\rightarrow$41$\rightarrow$26\\
		return r = 16
	}\\
	\pagebreak
	% 7.
	\item Exercise 17.3-1 (p495)\\
	\texttt{
	Left-Rotate(T,x)\\
	\tab y = x.right\\
	\tab x.right = y.left\\
	\tab if y.left $\neq$ T.NIL\\
	\tab\tab y.left.p = x\\
	\tab y.p = x.p\\
	\tab if x.p == T.NIL\\
	\tab\tab T.root = y\\
	\tab else if x == x.p.left\\
	\tab\tab x.p.left = y\\
	\tab else\\
	\tab\tab x.p.right = y\\
	\tab y.left = x\\
	\tab x.p = y\\
	\tab x.max = max(x.high, x.left.max, x.right.max)\\
	\tab y.max = max(y.high, y.left.max, y.right.max)\\
	}
	\pagebreak

	% 8.
	\item Problem 17-2 (p496)
	\begin{enumerate}
	\item \hphantom \\
	\texttt{
	Josephus(n,m)\\
	\tab head = new Node(1)\\
	\tab prev = head\\
	\tab for i = 2 to n\\
	\tab\tab newNode = new Node(i)\\
	\tab\tab prev.next = newNode\\
	\tab\tab prev = newNode\\
	\tab prev.next = head\\\\
	\tab while prev != prev.next\\
	\tab\tab for j = 1 to m-1\\
	\tab\tab\tab prev = prev.next\\
	\tab\tab print prev.next.key\\
	\tab\tab prev.next = prev.next.next\\\\
	\tab print prev.key
	}
	\item \hphantom \\
	\texttt{
	Josephus(n,m)\\
	\tab create order-statistic tree with keys 1...n\\
	\tab r = 0\\
	\tab for k = n downto 1\\
	\tab\tab r = (r + m - 1) mod k\\
	\tab\tab if r == 0\\
	\tab\tab\tab r = k\\
	\tab\tab x = Select(T.root, r)\\
	\tab\tab print x.key\\
	\tab\tab Delete(T, x)
	}
	\end{enumerate}
\end{enumerate}

\end{document}
